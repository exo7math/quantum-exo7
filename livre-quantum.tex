\documentclass[11pt]{report}
%\documentclass[10pt,twoside,openright]{report}

\usepackage[screen]{python}
%\usepackage[print]{python}


\begin{document}

%%%%%%%%%%%%%%%%%%%%%%%%%%%%%%%%%%%%%%%%%%%%%%%%%%%%%%%%%%%%%%%%%%
% Titre + préface + sommaire
\renewcommand{\contentsname}{Sommaire}

% Préface
\pagenumbering{roman}
\import{divers/}{preface_livre_quantum.tex}
\debutchapitres
\pagenumbering{arabic}

%%%%%%%%%%%%%%%%%%%%%%%%%%%%%%%%%%%%%%%%%%%%%%%%%%%%%%%%%%%%%%%%%%

\clearemptydoublepage

%======================================
% Partie 1 - Premiers pas quantiques
%======================================

\part{Premiers pas quantiques}

% 1 - Découverte
\import{decouverte/}{decouverte.tex}

% 2 - Utiliser un ordinateur quantique (avec Qiskit)
\import{ordinateur/}{ordinateur.tex}

% 3 - Nombres complexes
\import{complexes/}{complexes.tex}

% 4 - Vecteurs et matrices
\import{vecteurs/}{vecteurs.tex}

% 5 - Informatique classique
\import{classique/}{classique.tex}

% 6 - Physique quantique
\import{physique/}{physique.tex}

% 7 - Téléportation quantique
\import{teleportation/}{teleportation.tex}

%======================================
% Partie 2 - Algorithmes quantiques
%======================================

\part{Algorithmes quantiques}

% 8 - Un premier algorithme quantique
\import{algorithme/}{algorithme.tex}

% 9 - Portes quantiques
\import{portes/}{portes.tex}

% 10 - Algorithme de Deutsch--Jozsa
\import{deutsch/}{deutsch.tex}

% 11 - Algorithme de Grover
\import{grover/}{grover.tex}

%======================================
% Partie 3 - Algorithme de Shor
%======================================

\part{Algorithme de Shor}

% 12 - Arithmétique
\import{arithmetique/}{arithmetique.tex}

% 13 - Algorithme de Shor
\import{shor/}{shor.tex}

% 14 -Compléments d'arithmétique 
\import{complement/}{complement.tex}

% 15 - Transformée de Fourier discrète
\import{fourier/}{fourier.tex}

%======================================
% Partie 4 - Vivre dans un monde quantique
%======================================

\part{Vivre dans un monde quantique}

% 16 - Cryptographie quantique
\import{crypto/}{crypto.tex}

% 17 - Code correcteur
\import{code/}{code.tex}

% 18 - Avantage quantique
\import{avantage/}{avantage.tex}


\clearemptydoublepage

%======================================
% Postface et index
%======================================

% \import{annexe/}{annexe.tex}

\import{divers/}{postface_livre_quantum.tex}
%\newpage

\vfill
\bigskip
\bigskip

\mycenterline{Version 1.01 -- Avril 2024}



\end{document}

